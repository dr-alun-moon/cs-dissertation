%!TeX root=Dissertation.tex

\chapter{Some TeXnical Details}

\section{Structure of the file set}
The Terms of Reference and Dissertation are split into several files, to ease
editing, and exploit some of \LaTeX's capabilities.
The structure is relatively \emph{flat} with little hierarchy of directories,
directories and sub-directories can be added to simplify some of the structure
as the project grows.

The principle files are:

\paragraph{\texttt{Makefile}} I use \texttt{make} still as my build driver.
Any automated build system can work with \LaTeX\ files, it just needs to know
that PDFs are build from \texttt{.tex} files.

\paragraph{\texttt{TermsOfReference.tex}} This is the main file for creating
the \emph{Terms of Reference}.  It pulls in the \texttt{tor.tex} file, the
Gantt chart from \texttt{gantt.tex}, and the ethics and risk assessment forms
from scanned PDFs.  The document is typeset as an \texttt{article}.

\paragraph{\texttt{Dissertation.tex}} Is the main file for creating the
dissertation.   This pulls in a number of other files as required.	
It is typeset as a \texttt{book}.  The sectioning starts as \verb'\chapter'
and has \verb'\section'  within.  This way the terms of reference can be
included as a chapter in the appendix.
The chapters are each included using \verb'\include',  this allows the
chapters to be written as separate files.  The difference between
\verb'\include' and \verb'\input' is that \verb'\include' forces a new
right-hand page to start (good for starting chapters).

\subsection{Subsidiary files}
\paragraph{\texttt{tor.tex}} This file is the main \emph{Terms-of-Reference}
file.  By using \verb'\input' in the  \texttt{TermsOfReference.tex} document,
this gets included and typeset.  

\paragraph{\texttt{gantt.tex}} The Gantt chart is a little complex, so gets
put into a separate file, see section \ref{gantt-chart}.

\subsection{The Gantt Chart}\label{gantt-chart}
The Gantt chart in the Terms-of-Reference is drawn with another latex package.
It uses an \texttt{input} command to pull it into the tor (and dissertation).

Most of the file \path{Gantt.tex} is the setup of the Gant chart, in order to
make it fit in one page.  The bottom section is where you can define the
tasks.
Each bar has a title, a start date, and an end date.  Milestones have a title
and a date.  The \texttt{ms} option can be set to left or right to control
which side of the milestone mark the text label.

\begin{tcblisting}{listing only}
% --Tasks go here
% put in a title, a start date, end date...
 \ganttbar{TOR}{11/9/17}{13/10/17}
 \ganttbar[inline]{\emph{revise}}{15/10/17}{10/11/17}\\
 \ganttbar{Analysis}{18/9/17}{24/11/17}\\
 \ganttbar{Design}{31/10/17}{17/1/18}
 \ganttmilestone[ms=right]{Build complete (Obj \ref{write-code})}{18/1/18}\\
 \ganttbar{Exploration}{22/9/17}{15/11/17}
%
\end{tcblisting}
